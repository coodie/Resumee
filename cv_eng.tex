\documentclass[line]{res}
\usepackage[utf8]{inputenc}
\usepackage{polski}
\usepackage[polish]{babel}
\usepackage{enumitem}
\usepackage{setspace}
\usepackage{hyperref}

\begin{document}
\name{Przemysław Leśniak}
\address{ 
 Wrocław, przemek.lesniak1@gmail.com, \ tel.: 516 706 214}
 
\begin{resume}
\section{EDUCATION}

University of Wrocław, 2013 - now \\
Institute of Mathematics and Computer Science \\
\textbf{Degree course: Computer Science} \\
Estimated graduation year: July 2016 for Bachelor, July 2018 for Master's \\

Comprehensive secondary school No. V in Opole, 2010 - 2013 \\
\textbf{Matemathics-Informatics-Physics profile}

\section{EXPERIENCE}
	\textbf{Nokia C++ Summer Trainee: } July - September 2015 \\
	\\
	\textbf{Responsibilites } 
	\begin{itemize}
	\item{Implementation of parsing library in C++, inspired by Parsec from functional language Haskell. }
	\item{Implementation of parser for TTCN-3 language, using mentioned library}
	\end{itemize}
	\textbf{Gained skills}
	\begin{itemize}
	\item{Git revision control system}
	\item{C++ programming: C++11, template metaprogramming, STL, boost}
	\item{Google test unit testing framework }
	\item{Programming tools such as: GNU make, linux, bash, cmake, gcc, clang }
	\item{SQLite}
	\end{itemize}
	

\section{SKILLS}

\textbf{Programming languages and technologies}
\begin{itemize}
\item{\textbf{C++}: C++11, STL, templates, Google Test }
\item{\textbf{Haskell}: monads, applicative functors, type classes, Parsec }
\item{\textbf{Java}: Swing}
\item{\textbf{Python}: university experience }
\item{\textbf{Coq}: university experience }
\end{itemize}

\textbf{Others}
\begin{itemize}
\item{Ability to use revision control system \textbf{Git} }
\item{Strong problem solving, and analytical thinking abilities }
\item{Good knowledge of algorithms and data structures and ability to implement them}
\item{Good knowledge of functional and objective paradigm}
\item{Medium knowledge about \textbf{Linux}, and ability to use it as programming platform}
\item{Good knowledge of \textbf{english language} both written and spoken}
\end{itemize}
\pagebreak
\section{PROJECTS}
\textbf{hCompiler} \\
\url{https://github.com/coodie/hCompiler} \\
Simple compiler for subset of C. Written in Haskell, compiles directly to x86 assembly. \\

\begin{itemize}
\item{Handles while loops, conditionals, recursive functions, nested arithmetic expressions}
\item{Uses \textbf{GNU as} as back-end assembler and \textbf{gcc} to link}
\item{Has it's own parser written in \textbf{Parsec} }
\item{Extensive usage of monads, applicative functors and monad transformers}
\end{itemize}



\textbf{GraphDrawer} \\
\url{https://github.com/coodie/GraphDrawer} \\
Program for graph visualisation written in Java. Suited for competetive programming. \\

\begin{itemize}

\item{Uses special algorithm to draw graph on plane and make it 'eye-pleasing'}
\item{Can be used as real-time graph editor}
\item{Has lots of options: saving graph to image, performing some algorithms on graph  (for example: finding shortest paths)}
\item{Uses \textbf{Model-View-Controller} design pattern and \textbf{Swing} as GUI}
\end{itemize}

\textbf{Other projects}
\begin{itemize}
\item{MIPS assembler, written in Haskell }
\item{Program for visualising curves such as Bezier Curve, written in Haskell}
\item{Advanced sudoku game, written in Java }
\item{mmap based malloc, written in C}
\end{itemize}

\section{HOBBIES}
\begin{itemize}
\item{Programming contests}
\item{\textbf{Speedcubing}}
\item{\textbf{Popping} dance}
\item{ Electronic music }
\end{itemize}

\section{OTHERS}
Github's account \url{https://github.com/coodie/}\\
Codeforces's account \url{http://codeforces.com/profile/goovie}

\vfill

\end{resume}

\end{document}
