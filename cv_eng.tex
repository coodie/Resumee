
\documentclass[10pt,a4paper,sans]{moderncv}
\moderncvtheme{banking}
\usepackage[utf8]{inputenc}
\usepackage{polski}
\usepackage[polish]{babel}


\newcommand{\textbt}[1]{\textit{\textbf{#1}}}

\firstname{Przemysław}
\lastname{Leśniak}
\extrainfo{Computer Science Student}
\email{przemek.lesniak1@gmail.com}


\begin{document}
\maketitle

\section{Education}

\cventry{2013-2016}{BSc}{University of Wrocław}{Wrocław}{4.5}{Computer Science}{}
\cventry{2016-now}{MSc}{University of Wrocław}{Wrocław}{}{Computer Science}{}

\section{Experience}
\cventry{2016.07-now}{Software Engineer}{Nokia}{Wrocław}{}
{Development of TTCN-3 compiler written in C++}
\cventry{2015.07-2015.10}{C++ Summer Trainee}{Nokia}{Wrocław}{}{Parsing library project in C++}
\section{Skills}

\cvitem{Programming languages}{C, C++, Haskell, Java, Python}
\cvitem{Tools}{Linux, git, bash, vallgrind, callgrind }{}
\cvitem{Others}{Algorithms, Operating systems, Functional Programming}
\cvitem{Languages}{Proficiency in English language}

\section{Project Highlights}
\cvitem{hCompiler}{Compiler for subset of C written in Haskell. Compiles directly to x86 assembly}
\cvitem{GraphDrawer}{Graph visualisation tool written in Java. Aimed for competitive programming.}
\cvitem{mimiker} {Implementation of virtual memory for mimiker operating system: paging, TLB management, virtual memory mapping. Used C and MIPS assembly.}
\cvitem{CubeSolver}{Program for Rubik's cube solving written in Python. }

\section{Other}
\cvitem{\url{https://github.com/coodie/}}{contains codes to all highlighted projects}
\cvitem{\url{http://codeforces.com/profile/goovie}}{site with algorithmic contests }
\cvitem{Hobbies}{Popping, Speedcubing, Psychology}
\end{document}
