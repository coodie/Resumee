\documentclass[]{twentysecondcv}
\usepackage{polski}
\usepackage[polish]{babel}
\usepackage[utf8]{inputenc}
 
\begin{document}

%%%%%%%%%%%%%%%%%
%%PROFILE SIDE BAR%%
%%%%%%%%%%%%%%%%%

%%%%%%%%%%%%%%%%
%%PERSONAL INFO%%%
%%%%%%%%%%%%%%%%

\profilepic{icon_cv.png} %path of profile pic
\cvname{Przemysław Leśniak} %your name
\cvjobtitle{Computer Science student}%your actual job position
\cvdate{7 May 1994}%date of birth
\cvaddress{Saarbrücken, Germany}%address
\cvmail{przemek.lesniak1@gmail.com}%e-mail
\cvnumberphone{+48 516706214}%telphone number


% \cvsite{http://en.wikipedia.org}%personal site

\aboutme{I am a passionate Computer Science student from Poland that enjoys problem solving, programming and figuring out how things work. Currently on student exchange in Saarbrücken, Germany.
}%About me section

%%%%%%%%%%%%%%%%%%%%%%%%%%%%%%%%%%%%%%%%%%%%%%%%%%%%%%%%%%%%%%
%%%%%%Skill bar section, each skill must have a value between 0 an 6 (float)%%%%%%%
%%%%%%%%%%%%%%%%%%%%%%%%%%%%%%%%%%%%%%%%%%%%%%%%%%%%%%%%%%%%%%
\skills{English language/5.0, git/5.0, Object Oriented Programming/3.0, Linux/3.0, Haskell/2.0, Java/2.5, Python/2.5, C/3.5, C++/4.5}

%%%%%%%%%%%%%%%%%%%%%%%%%%%%%%%%%%%%%%%%%%%%%%%%%%%%%%%%%%%%%%
%%%%%%Skill text section, each skill must have a value between 0 an 6%%%%%%%%%%%%
%%%%%%%%%%%%%%%%%%%%%%%%%%%%%%%%%%%%%%%%%%%%%%%%%%%%%%%%%%%%%%

\makeprofile
%%%%%%%%%%%%%%%%%%%%
%%END PROFILE SIDE BAR%%
%%%%%%%%%%%%%%%%%%%%

%%%%%%%%%%%%%%%%%%%%
%%%%%%%%BODY%%%%%%%%
%%%%%%%%%%%%%%%%%%%%

%%%%%%%%%%%%%%%%%%%%
%%SIMPLE SECTION%%%%%%
%%%%%%%%%%%%%%%%%%%%

\section{education}

%%%%%%%%%%%%%%%%%%%%%%%%%%%%%%%%%%%%%%%%%%
%%%%%%%%%%%%%TWENTY LIST ITEMS%%%%%%%%%%%%%%
%%    Four arguments: date; title; where; description %%%%
%%%%%%%%%%%%%%%%%%%%%%%%%%%%%%%%%%%%%%%%%%
\begin{twenty}
  \twentyitem
    {since 2017}
    {M.Sc.}
    {Saarland University}
    {Computer Science, one semester student exchange}
  \twentyitem
    {since 2016}
    {M.Sc.}
    {University of Wrocław}
    {Computer Science}
  \twentyitem
    {2013-2016}
    {B.Sc.}
    {University of Wrocław}
    {Computer Science: 4.5/5.0
    
    \emph{Virtual memory subsystem for mimiker operating system} }
\end{twenty}


%%%%%%%%%%%%%%%%%%%%%%%%%%%%%%%%%%%%%%%%%%
%%%%%%%%%TWENTY LIST SHORTITEMS%%%%%%%%%%%%%%
%%% Two arguments: date; title/description %%%%%%%%%%
%%%%%%%%%%%%%%%%%%%%%%%%%%%%%%%%%%%%%%%%%%

\section{experience}

\begin{twenty}
  \twentyitem
    {2017}
    {Google Summer of Code}
    {Remote work}
    {
    \emph{Improving LLVM Backend for Chapel Compiler}
    
    - Improved vectorization by fixing a serious bug and adding extra metadata in LLVM IR which in some cases improved performance of executed code by 400\%. 
    
    }
  \twentyitem
    {2016-2017}
    {Nokia, C++ Software Engineer}
    {Wrocław}
    {
    \emph{TTCN-3 Compiler Project}
    
     - Greatly reduced number of memory allocations in compiled code using object pool-like design pattern inspired by slab allocator leading to 20\% performance gain on average.
     
     - Reduced number of copy operations by introducing move operation in runtime and adding it to compiler code generation that resulted in 10\% performance gain in some cases. }
     
\twentyitem
    {2015}
    {Nokia, C++ Summer Trainee}
    {Wrocław}
    {
    \emph{Parsing library project}
    
    - Participated in library design inspired by Parsec library from Haskell language that was used to implement partial parser for TTCN-3 language.
    
    - Designed and implemented algorithm (based on pushdown automata) to locate changes in code in real time that would need to be re-parsed.
    
    - Integrated the algorithm and the parser into QtCreator to provide IDE functionality like auto-completion and jumping to function definitions.
    
    }

\end{twenty}

\section{project highlights}
\begin{twenty}
	\twentyitem
	{mimiker}
	{University of Wrocław operating system}
	{C, MIPS assembly}
	{ Played a big role in virtual memory subsystem, mutex implementation, gdb scripting, ramdisk loading, basic filesystems. Helped other students get into the project. }
	\twentyitem
	{quant}
	{Lossy Image Compression}
	{C++14}
	{Reduces image size by 80\% while preserving good image quality.
	Optimized typically slow algorithm by using tuned data structures and parallelizing parts of code.
	}
	
	\twentyitem
	{hCompiler}
	{Compiler that compiles tiny subset of C}
	{Haskell}
	{Compiles directly to x86 assembly using syntax directed code generation. Supports working recursion and basic language constructs. }
	
	\twentyitem
	{GraphDrawer\ \ \ }
	{Visual and real-time editing graph drawing program}
	{Java}
	{Rich in functionality: performs various algorithms on graph, saves graphs as images, drawes pretty graph images.}
	
	\twentyitem
	{CubeSolver}
	{Rubik's cube solving program}
	{Python}
	{Finds solution to physical rubik's cube and guides the user through it. }
	
\end{twenty}


\section{other information}
Hobbies: popping dance, speedcubing

Github: \url{https://github.com/coodie/}

CodeForces: \url{http://codeforces.com/profile/goovie}


%%%%%%%%%%%%%%%%%%%%
%%%%%ENDBODY%%%%%%%%
%%%%%%%%%%%%%%%%%%%%

\end{document} 