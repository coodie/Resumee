\documentclass[]{twentysecondcv}
\usepackage{polski}
\usepackage[polish]{babel}
\usepackage[utf8]{inputenc}
 
\begin{document}

%%%%%%%%%%%%%%%%%
%%PROFILE SIDE BAR%%
%%%%%%%%%%%%%%%%%

%%%%%%%%%%%%%%%%
%%PERSONAL INFO%%%
%%%%%%%%%%%%%%%%

\profilepic{icon_cv.png} %path of profile pic
\cvname{Przemysław Leśniak} %your name
\cvjobtitle{Computer Science student}%your actual job position
\cvdate{7 may 1994}%date of birth
\cvaddress{Saarbruecken, Germany}%address
\cvnumberphone{+48 516706214}%telphone number
\cvmail{przemek.lesniak1@gmail.com}%e-mail

% \cvsite{http://en.wikipedia.org}%personal site


\aboutme{
I am passionate Computer Science student from Poland that enjoys problem solving and 
technical things. Currently on student exchange in Saarbruecken, Germany.
}%About me section

%%%%%%%%%%%%%%%%%%%%%%%%%%%%%%%%%%%%%%%%%%%%%%%%%%%%%%%%%%%%%%
%%%%%%Skill bar section, each skill must have a value between 0 an 6 (float)%%%%%%%
%%%%%%%%%%%%%%%%%%%%%%%%%%%%%%%%%%%%%%%%%%%%%%%%%%%%%%%%%%%%%%
\skills{git/5.0, Object Oriented Programming/3.0, Algorithms/4.0, C++/4.5, English/5.0}

%%%%%%%%%%%%%%%%%%%%%%%%%%%%%%%%%%%%%%%%%%%%%%%%%%%%%%%%%%%%%%
%%%%%%Skill text section, each skill must have a value between 0 an 6%%%%%%%%%%%%
%%%%%%%%%%%%%%%%%%%%%%%%%%%%%%%%%%%%%%%%%%%%%%%%%%%%%%%%%%%%%%
\skillstext{{Java/2.5}, {Python/3.0}, {C/4.0}, {Haskell/2.0}, {Coq/1.5}, {Assembly/2.0}, Linux Command Line/3.0}

\makeprofile
%%%%%%%%%%%%%%%%%%%%
%%END PROFILE SIDE BAR%%
%%%%%%%%%%%%%%%%%%%%

%%%%%%%%%%%%%%%%%%%%
%%%%%%%%BODY%%%%%%%%
%%%%%%%%%%%%%%%%%%%%

%%%%%%%%%%%%%%%%%%%%
%%SIMPLE SECTION%%%%%%
%%%%%%%%%%%%%%%%%%%%

\section{education}

%%%%%%%%%%%%%%%%%%%%%%%%%%%%%%%%%%%%%%%%%%
%%%%%%%%%%%%%TWENTY LIST ITEMS%%%%%%%%%%%%%%
%%    Four arguments: date; title; where; description %%%%
%%%%%%%%%%%%%%%%%%%%%%%%%%%%%%%%%%%%%%%%%%
\begin{twenty}
  \twentyitem
    {since 2017}
    {M.Sc.}
    {Saarland University}
    {Computer Science, one semester student exchange}
  \twentyitem
    {since 2016}
    {M.Sc.}
    {University of Wrocław}
    {Computer Science}
  \twentyitem
    {2013-2016}
    {B.Sc.}
    {University of Wrocław}
    {Computer Science: 4.5/5.0
    
    \emph{Virtual memory subsystem for mimiker operating system} }
\end{twenty}


%%%%%%%%%%%%%%%%%%%%%%%%%%%%%%%%%%%%%%%%%%
%%%%%%%%%TWENTY LIST SHORTITEMS%%%%%%%%%%%%%%
%%% Two arguments: date; title/description %%%%%%%%%%
%%%%%%%%%%%%%%%%%%%%%%%%%%%%%%%%%%%%%%%%%%

\section{experience}

\begin{twenty}
  \twentyitem
    {2017}
    {Google Summer of Code}
    {Remote}
    {
    \emph{Improving LLVM Backend for Chapel Compiler}
    
    The biggest acomplishments of the project was improving vectorization, which in some cases improved performance of executed code by 400\%. }
  \twentyitem
    {2016-2017}
    {Nokia, C++ Software Engineer}
    {Wrocław}
    {
    \emph{Improving performance of code generated by TTCN-3 compiler}
    
     Main approach to improving performance was reducing memory allocations using Object Pool technique inspired by Slab Allocator. In the end average performance was improved by 20\%, in extreme cases even by 80\%. }
\twentyitem
    {2015}
    {Nokia, C++ Summer Trainee}
    {Wrocław}
    {
    \emph{Parsing library project}
    
     Library design was inspired by Parsec library from Haskell language. Library was used to implement parser for TTCN-3 language and later integrate it to QtCreator to have general IDE functionality, like auto completion and jumping to function definitions. }

\end{twenty}

\section{project highlights}
\begin{twenty}
	\twentyitem
	{}
	{mimiker}
	{}
	{ Operating system being developed on university. My work involved working on virtual memory subsystem (TLB, page allocation, virtual page mapping), mutex implementation, gdb scripting, ramdisk loading, basic filesystems. }
	\twentyitem
	{}
	{quant}
	{}
	{Lossy Image Compression based on vector quantization. Achieves good compression ratios and image quality.}
	\twentyitem
	{}
	{hCompiler}
	{}
	{Compiler that compiles tiny subset of C language written in Haskell. Compiles directly to x86 assembly using syntax directed code generation. Has working recursion. }
	\twentyitem
	{}
	{Lossy Image Compression}
	{}
	{Lossy Image Compression based on vector quantization. Achieves good compression ratios and image quality.}

		\twentyitem
	{}
	{Lossy Image Compression}
	{}
	{Lossy Image Compression based on vector quantization. Achieves good compression ratios and image quality.}
\end{twenty}


\section{other information}
Hobbies: Popping, Speedcubing

Github: \url{https://github.com/coodie/}

CodeForces: \url{http://codeforces.com/profile/goovie}


%%%%%%%%%%%%%%%%%%%%
%%%%%ENDBODY%%%%%%%%
%%%%%%%%%%%%%%%%%%%%

\end{document} 