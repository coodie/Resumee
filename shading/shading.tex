%%% ====================================================================
%%%  @LaTeX-file{
%%%     filename        = "shading.tex",
%%%     version         = "1.1",
%%%     date            = "29 September 1992",
%%%     time            = "08:30:45 MDT",
%%%     author          = "Art Mulder",
%%%     address         = "Department of Computing Science
%%%                        615 General Services Building
%%%                        University of Alberta
%%%                        Edmonton, Alberta
%%%                        T6G 2H1, Canada.",
%%%     telephone       = "n/a",
%%%     FAX             = "(403) 492-1071",
%%%     checksum        = "49623 227 1018 8613",
%%%     email           = "art@cs.ualberta.ca",
%%%     codetable       = "ISO/ASCII",
%%%     keywords        = "documentation",
%%%     supported       = "yes",
%%%     docstring       = "The LaTeX document briefly describes the
%%%                        LaTeX style file ``shading.sty''.
%%%                        Shading.sty is a style file for putting
%%%                        text on a shaded background in LaTeX.
%%%
%%%                        Requires a Postscript printer and
%%%                        dvi converter",
%%%  }
%%% ====================================================================
\documentstyle[12pt, moretext, shading]{article}
%%%%%%%%%%%%%%%%%%%%%%%%%%%%%%%%%%%%%%%%%%%%%%%%%%%%%%%%%%%%%%%%%%%%%%%%%%%%%
% shading.tex                           Art Mulder ( art@cs.ualberta.ca )
% 25 September, 1992
%
% A brief manual on the use of the "shading.sty" LaTeX style file.
%%%%%%%%%%%%%%%%%%%%%%%%%%%%%%%%%%%%%%%%%%%%%%%%%%%%%%%%%%%%%%%%%%%%%%%%%%%%%
% Usage:
%   - Process this file with latex.  (The file "shading.sty" must either
%     be in a standard location with the other LaTeX style files, or in
%     the same directory as this file).
%   - Convert to postscript, via a tool such as "dvips"
%   - either view with a Postscript previewer (like ghostview) or
%     print to a postscript printer.
%
% In the above \documentstyle command, the style file ``moretext'' is
% optional.  It simply shrinks the margins to something that I like a
% bit better than the default values for article.
%%%%%%%%%%%%%%%%%%%%%%%%%%%%%%%%%%%%%%%%%%%%%%%%%%%%%%%%%%%%%%%%%%%%%%%%%%%%%
% Help yourself to whatever you want from this file.
% Caveat Emptor.
%%%%%%%%%%%%%%%%%%%%%%%%%%%%%%%%%%%%%%%%%%%%%%%%%%%%%%%%%%%%%%%%%%%%%%%%%%%%%

\begin{document}

%       Title "Page"
%---------------------------------------------------------------------------
\title{The {\tt shading.sty} document style}
\author{Arthur E. Mulder\thanks{Author of this document only.  The
  style file itself has had several authors / contributors:
  J\'{e}r\^{o}me Maillot ({\tt maillot@bora.inria.fr}),
  Leo ??  ({\tt LEO@vaxc.cc.monash.edu.au}), and
  M.A.R. ({\tt mroth@afit.af.mil}).}
\date{29 September 1992 \\
Version 1.01}
\maketitle

%---------------------------------------------------------------------------

\section{Introduction}
  The purpose of this style file is to permit you, in LaTeX, to place text
  in a box with a shaded background.  This particular solution uses
  Postscript to do so, and is therefore of no use to you if you do not have
  a postscript printer.  Furthermore, if you wish to be able to preview your
  document before printing, you will also need a Postscript viewer, such as
  {\tt Ghostview}.

%---------------------------------------------------------------------------
\section{Usage}
  To use this style option, you must add the name of the style file, {\tt
  shading} to the {\tt documentstyle} command in your document:

  \begin{quote}
        \verb|\documentstyle[... shading ...]{article}|
  \end{quote}

  Note that this should also work with teh {\tt report} and {\tt book}
  styles.  You are not restricted to using the {\tt article} style.

\subsection{Shading Commands}
  There are two commands provided by this style:

  \begin{quote}
    \verb|\textshade|[{\em grayscale}]\{{\em corneroption}\}\{{\em text
        to be shaded}\}

    \verb|\parashade|[{\em grayscale}]\{{\em corneroption}\}\{{\em paragraph
        to be shaded}\}
  \end{quote}


  \begin{tabular}{lp{4.5in}}
  {\tt grayscale} & This is a number from 0 to 1.  The higher the number,
                  the {\em lighter} the resulting shading.  With a value
                  of 1, there is essentially no shading, and you end up
                  with just a box.  A value of 0 is probably of little
                  practical use.  If this value is omitted, a default
                  value of .95 is used. \\
  {\tt corneroption} &   This is either the {\tt sharpcorners} or
                    {\tt roundcorners}, with obvious effects upon the box
                    that is drawn. \\
  \end{tabular}


\subsection{Limitations}
  There are a number of limitations that go along with this style:

  \begin{itemize}

    \item You can not break text across lines within the \verb|\textshade|
      command.

    \item You can not break text across pages with either command.

    \item You can not specify the line width of the surrounding box, or
      the amount of space between the surrounding box and the text to
      be shaded {\em as arguments}.

      See the examples below for how you {\em can}
      fiddle with the line width of the surrounding box.

    \item Modifying the xgrayspace and ygrayspace in the argument
      will only affect the space added to the right and
      bottom of the box, respectively.

  \end{itemize}

%---------------------------------------------------------------------------
\section{Examples of Usage}

%
%% 1)
%
\subsection{}
  \noindent\begin{minipage}{3in}
    This is a test of a\ \textshade[1]{roundcorners}{shaded box} routine.
    With a grayscale of 1, we essentially get no shading, and just a box.
  \end{minipage}\hspace{.25in}\begin{minipage}{3in}
    \begin{verbatim}
This is a test of a\
\textshade[1]{roundcorners}{shaded box}
routine.  With a ...
     \end{verbatim}\end{minipage}

%
%% 2)
%
\subsection{}
  \noindent\begin{minipage}{3in}
    This is another test of a\ \textshade{sharpcorners}{shaded box}
    routine.  The default shading of .95 is used here.  Note also the
    difference between a ``sharpcorners'' and a ``roundcorners'' box.
  \end{minipage}\hspace{.25in}\begin{minipage}{3in}
\begin{verbatim}
...
\textshade{sharpcorners}{shaded box}
...
\end{verbatim}
\end{minipage}

%
%% 3)
%
\subsection{}
\noindent\begin{minipage}{3in}
    \parashade[.995]{roundcorners}{\gdef\outlineboxwidth{.5}%
    This is one very long line which I expect will be broken over one or more
    lines. The idea is to have this paragraph enclosed in a shaded box. I'll
    just keep on typing until I can be sure that there are more than two lines
    in this paragraph.
    }
 \end{minipage}\hspace{.25in}
\begin{minipage}{3in}

\begin{verbatim}
\parashade[.995]{roundcorners}
{\gdef\outlineboxwidth{.5}%
This is one very ... paragraph.}
\end{verbatim}\end{minipage}

  \smallskip
    In the prior example, take note of how the width of the box surrounding
    the paragraph was changed from it's default of 2 (points?) to .5,
    through the use of the \verb|\gdef\outlineboxwidth{.5}| command.

    This is probably a pretty kludgey way of doing it, but it works well
    enough for now.  If you feel like fixing it up, you're welcome to give
    it a try!

%
%% 4)
%
\subsection{}
  \noindent\begin{minipage}{3in}
    So, let's try fiddling with the linewidth to get
    \textshade[.8]{sharpcorners}{\gdef\outlineboxwidth{0}no box}
    around this text.  Not bad, eh?
  \end{minipage}\hspace{.25in}\begin{minipage}{3in}
\begin{verbatim}
...
\textshade[.8]{sharpcorners}
{\gdef\outlineboxwidth{0}no box}
...
\end{verbatim}\end{minipage}

%---------------------------------------------------------------------------
\section{Moving On\ldots}
  For further information about ways that you can tweak this style to serve
  you better, please see the source code for the style:  {\tt shading.sty}

%---------------------------------------------------------------------------
\section{Caveat Emptor}
  I did not write or design this document style option.  I simply wrote this
  document, first as an exercise in trying out the shading document style,
  and secondly as a service to any others whom I might pass this on to.
  A fair amount of the content of this file came from the contents of the
  style file {\tt shading.sty}. \hfill {\em \ldots Art Mulder, 9/25/92}
%---------------------------------------------------------------------------
\end{document}
